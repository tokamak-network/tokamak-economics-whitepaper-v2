\section*{Summary}

Ethereum was founded to establish a decentralized and censorship-resistant global computer, prioritizing trustlessness and security above all else. However, maintaining this level of decentralization inherently limits transaction throughput on the mainnet. To overcome this scalability constraint while preserving Ethereum’s robust security, the industry has adopted Layer 2 rollups as the standard infrastructure. This architectural progression allows execution to occur off-chain, ensuring that the mainnet remains the security anchor for correctness and safety.

As this Layer 2 ecosystem matures, distinct needs for customization and sovereignty are driving the market beyond general-purpose solutions. Developers and enterprises increasingly require dedicated execution environments that offer flexibility and ownership over their infrastructure. This demand is catalyzing the rise of small rollups, which are agile, application-specific networks tailored to unique business requirements rather than sharing a monolithic state. We identify this shift toward modular, customized chains as the next major phase of blockchain utility.

Nevertheless, this fragmentation introduces a critical security crisis. While deploying a custom rollup is technically accessible, establishing a decentralized and reliable verification system remains resource-intensive. Small and independent rollups often struggle to attract a sufficient number of validators and liquidity providers. Consequently, these networks are disproportionately exposed to the \textit{verifier's dilemma}, where the economic incentives for honest monitoring are insufficient, leaving them vulnerable to censorship or operational failure.

Tokamak Network resolves this structural problem by providing a systemized verification economics layer. We offer a unified platform where independent small rollups can inherit institutional-grade security without the burden of bootstrapping it alone. At the core is the TON token, serving as the universal asset for security bonds, gas, and governance. Through mechanisms such as the Randomized Attention Test (RAT), we align the incentives of validators to ensure continuous and high-integrity monitoring across the entire network, regardless of the size of an individual rollup.

To guarantee long-term stability, this whitepaper introduces \textit{TON Staking V3}, a performance-driven distribution model. Unlike traditional systems that reward passive capital, V3 allocates protocol seigniorage based on effective economic activity, specifically measured by Bridged TON. This structure ensures that the value of TON grows in tandem with the collective success of the rollup ecosystem, establishing a sustainable economic blueprint for a secure and decentralized future.